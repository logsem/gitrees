We aim to provide labeled transition systems corresponding
to visible effects happening in programs written in guarded interaction
trees. We pursue two aims. First, we want to study these LTSs as they are.
Second, we want to have an ability to hide some effects in case they are
invisible to external observers. Last, we want to be able to relate
two or more LTSs corresponding to different programs to argue
that they simulate each other up to possible hidden effects.
Thus, we can study security properties of programs, and, moreover,
obtain stronger reasoning principles for relating programs.

\begin{figure}
  \begin{align*}
    E_{nondet} &\eqdef \{\mathtt{flip}\} \\
    \Ins_{\mathtt{flip}}(X) &\eqdef \Tunit \\
    \Outs_{\mathtt{flip}}(X) &\eqdef \Tbool \\
    \mathtt{br}~x~y & \eqdef \Vis_{\mathtt{flip}}((), \Lam b. \IF b \{ x \} \{ y \}) \\
    r_{\mathtt{flip}}((),b\vec{b},\kappa) &= \Some(\kappa~b, \vec{b})
  \end{align*}
  \caption{state is a coinductive list of booleans}
\end{figure}

\begin{align*}
  \mathsf{spin} \cceq \MU a \col \latert \IT. \Tau(a) \equiv \Tick(\mathsf{spin})
\end{align*}
\begin{align*}
  \mathsf{p} \cceq \WHILE{\true}{\OUTPUT~0} \equiv \OUTPUT~0 \SEQ \WHILE{\true}{\OUTPUT~0} \equiv \dots
\end{align*}
\begin{align*}
  \mathsf{q} \cceq \mathsf{br}~\mathsf{spin}~\mathsf{p}
\end{align*}

\begin{align*}
  (\alpha, \sigma) \prec^{\tau} (\beta, \sigma') & \cceq \alpha = \Tick(\beta) \wedge \sigma = \sigma' \\
  (\alpha, \sigma) \prec^{i_{x}} (\beta, \sigma') & \cceq \Exists i\,x\,k. \alpha = \Vis_i(x,k)
    \wedge \reify(\alpha,\sigma) = (\Tick(\beta),\sigma')
\end{align*}

\begin{figure}
  \begin{tikzpicture}
    \node[state] (1) {$\mathsf{q}$};
    \node[state, below left of=1] (2) {$\mathsf{p}$};
    \node[state, right of=2] (3) {$\mathsf{spin}$};
    \draw (2) edge[above] node{$\mathsf{flip}_{\mathsf{false}}$} (1)
    (3) edge[loop right] node{$\tau$} (3)
    (1) edge[loop right] node{$\OUTPUT~0$} (1)
    (2) edge[below] node{$\mathsf{flip}_{\mathsf{true}}$} (3);
  \end{tikzpicture}
  \caption{reifiers are the same}
\end{figure}

\begin{figure}
  \begin{tikzpicture}
    \node[state] (1) {$\mathsf{q}$};
    \node[state, below left of=1] (2) {$\mathsf{p}$};
    \node[state, right of=2] (3) {$\mathsf{spin}$};
    \draw (2) edge[above] node{$\tau$} (1)
    (3) edge[loop right] node{$\tau$} (3)
    (1) edge[loop right] node{$\OUTPUT~0$} (1)
    (2) edge[below] node{$\tau$} (3);
  \end{tikzpicture}
  \caption{some effects can be marked as hidden}
\end{figure}

\begin{figure}
  \begin{tikzpicture}
    \node[state] (1) {$\mathsf{q}$};
    \node[state, below left of=1] (2) {$\mathsf{p}$};
    \draw (2) edge[above] node{$\OUTPUT~0$} (1)
    (1) edge[loop right] node{$\OUTPUT~0$} (1);
  \end{tikzpicture}
  \caption{wip}
\end{figure}

%%% Local Variables:
%%% mode: latex
%%% TeX-master: "../main"
%%% End:
